% ==========================================
% BAB II STUDI LITERATUR
% ==========================================
\chapter{STUDI LITERATUR}
\label{chap:studi-literatur}
\section{Sistem Tiket Elektronik (\textit{E-Ticket})}
\label{sec:eticket}

Perkembangan teknologi informasi telah mengubah paradigma layanan di berbagai sektor industri, termasuk dalam manajemen akses dan reservasi melalui sistem tiket elektronik atau \textit{e-ticket}. Subbab ini akan membahas definisi, evolusi, serta karakteristik fundamental dari sistem \textit{e-ticket}.

\subsection{Definisi dan Konsep Dasar}
Menurut Kamus Besar Bahasa Indonesia (KBBI), tiket atau karcis adalah surat kecil (carik kertas khusus) sebagai tanda telah membayar ongkos dan sebagainya (untuk naik bus, menonton bioskop, dan sebagainya). Tiket merupakan dokumen yang berfungsi sebagai hak akses atau tanda pembayaran yang sah untuk menggunakan suatu layanan. Seiring dengan perkembangan teknologi, terjadi transformasi bentuk tiket konvensional yang berbasis kertas menjadi wujud digital yang tersimpan dalam basis data komputer, yang disebut sebagai \textit{electronic ticket} atau \textit{e-ticket}.

Secara konseptual, \textit{e-ticket} bukan sekadar penggantian media kertas, melainkan sebuah kontrak digital yang merepresentasikan hak kepemilikan atas suatu layanan atau produk. Informasi yang sebelumnya tercetak di atas kertas seperti detail acara, nomor kursi, dan identitas pemegang, kini dikodekan menjadi data digital yang dihubungkan dengan basis data di server pusat. Hal ini memungkinkan proses validasi dilakukan secara \textit{real-time} melalui pencocokan data, bukan sekadar pemeriksaan visual fisik kertas.

\subsection{Evolusi dan Transformasi Digital}
Pergeseran menuju \textit{e-ticket} merupakan bagian dari proses inovasi layanan yang lebih luas. Dalam konteks transportasi publik, \textcite{lubeck2012electronic} menjelaskan bahwa tiket elektronik dikembangkan sebagai evolusi dari sistem kartu pita magnetik dan tiket kertas konvensional. Pengembangan ini didorong oleh kekhawatiran akan inefisiensi dalam manajemen informasi dan kontrol operasi pada sistem terdahulu.

Pada fase awal, sistem konvensional seringkali terkendala oleh keterbatasan dalam pelacakan data. Adopsi sistem teknis terkomputerisasi kemudian muncul sebagai solusi untuk meningkatkan efisiensi dan efektivitas operasional. Menurut \textcite{lubeck2012electronic}, implementasi sistem tiket elektronik merupakan bentuk inovasi proses yang merampingkan dan mengkualifikasi operasional dengan mengurangi proses manual sehingga meningkatkan kualitas layanan secara keseluruhan. Transformasi ini mengubah cara pengelolaan informasi karena sistem kini mampu meregistrasi pengguna, mengontrol penjualan kredit, dan menerbitkan laporan manajemen yang akurat untuk pemantauan data.

\subsection{Keunggulan dan Efisiensi Operasional}
Adopsi luas sistem \textit{e-ticket} didorong oleh berbagai keunggulan signifikan dibandingkan sistem konvensional. \textcite{chen2007passenger} menyoroti bahwa motivasi utama maskapai penerbangan beralih ke \textit{e-ticketing} adalah penghematan biaya distribusi tiket dan biaya penanganan (\textit{handling overheads}). Sistem ini memungkinkan eliminasi tiket kertas, yang berdampak langsung pada pengurangan biaya tenaga kerja, pencetakan, pengiriman, dan akuntansi. Bagi pengguna, manfaat utamanya adalah kenyamanan akibat sifat tiket yang \textit{paperless}, yang secara spesifik menghilangkan risiko kehilangan tiket fisik sebelum perjalanan.

Di sisi lain, dalam konteks transportasi darat, \textcite{lubeck2012electronic} menekankan bahwa keuntungan krusial dari \textit{e-ticket} terletak pada peningkatan manajemen informasi dan kontrol. Sistem ini efektif membatasi perdagangan tiket ilegal (\textit{illegal trade}) yang sebelumnya marak terjadi pada tiket fisik, serta mempersulit penyalahgunaan manfaat tiket khusus (seperti tiket pelajar) karena kredit tiket kini bersifat personal dan tidak dapat dipindahtangankan. Selain itu, sistem elektronik juga meningkatkan keamanan dengan mengurangi jumlah uang tunai yang beredar di dalam kendaraan, sehingga mengurangi daya tarik bagi tindak kejahatan seperti perampokan. 

\section{Teknologi \textit{Quick Response} (QR) Code}
\label{sec:qrcode}

\subsection{Sejarah dan Prinsip Kerja}
\textit{Quick Response Code} (QR Code) adalah jenis kode batang matriks dua dimensi yang dikembangkan oleh Denso Wave pada tahun 1994. Awalnya ditujukan untuk pelacakan inventaris suku cadang kendaraan, teknologi ini kini telah diadopsi secara masif di berbagai sektor mulai dari pemasaran hingga manajemen akses \cite{tiwari2016introduction, shin2012psychology}. \textcite{shin2012psychology} mendefinisikan QR Code sebagai pola persegi yang terdiri dari modul hitam dengan latar belakang putih yang dirancang untuk didekodekan dengan kecepatan tinggi menggunakan perangkat pemindai atau kamera ponsel pintar.

Berbeda dengan kode batang (\textit{barcode}) satu dimensi yang hanya menyimpan data secara horizontal, QR Code mengodekan informasi dalam dua arah, yaitu vertikal dan horizontal. Struktur dua dimensi ini memungkinkan QR Code memiliki densitas informasi yang jauh lebih tinggi dan kapasitas penyimpanan yang lebih besar dalam ruang fisik yang lebih kecil dibandingkan pendahulunya \cite{alsuhibany2025innovative}. Kapasitas ini memungkinkan penyimpanan berbagai jenis mode data, termasuk numerik, alfanumerik, biner, hingga karakter Kanji \cite{tiwari2016introduction}, yang menjadikannya medium ideal untuk menyimpan data tiket elektronik yang kompleks.

\subsection{Struktur QR Code}
Kemampuan QR Code untuk dibaca dengan cepat dan akurat (\textit{high-speed reading}) didukung oleh strukturnya yang unik. Berdasarkan spesifikasi teknis yang dijelaskan oleh \textcite{tiwari2016introduction}, setiap simbol QR Code dibangun dari modul-modul persegi yang disusun dalam \textit{array} persegi reguler. Struktur ini terdiri dari dua bagian utama, yaitu pola fungsi (\textit{function patterns}) dan wilayah pengodean (\textit{encoding region}), yang dikelilingi oleh batas zona tenang (\textit{quiet zone}) di keempat sisinya.

Wilayah pengodean (\textit{encoding region}) berisi data yang merepresentasikan informasi versi, informasi format, data konten, dan \textit{codeword} koreksi kesalahan. Sementara itu, pola fungsi adalah bentuk-bentuk spesifik yang ditempatkan di area tertentu untuk memastikan pemindai dapat mengidentifikasi dan mengorientasikan kode dengan benar. Terdapat empat jenis pola fungsi, yaitu \textit{finder pattern}, \textit{separator}, \textit{timing patterns}, dan \textit{alignment patterns}.

Komponen visual utama QR Code dapat dilihat pada Gambar \ref{fig:struktur_qrcode}, dan untuk rincian dari \textit{function patterns} dijelaskan sebagai berikut:

\begin{figure}[h]
    \centering
    \includegraphics[width=0.8\textwidth]{image/struktur_qrcode.png} 
    \caption{Struktur QR Code \cite{tiwari2016introduction}}
    \label{fig:struktur_qrcode}
\end{figure}

\begin{enumerate}[a)]
    \item \textbf{\textit{Finder Pattern}: }
    Tiga struktur kotak konsentris yang terletak di sudut kiri atas, kanan atas, dan kiri bawah. Pola ini memungkinkan pemindai mendeteksi posisi dan orientasi kode dari segala arah (360 derajat), sehingga pemindaian dapat dilakukan secara omni-direksional.
    
    \item \textbf{\textit{Separators}: }
    Area selebar satu modul berwarna putih (kosong) yang terletak di antara setiap \textit{finder pattern} dan wilayah pengodean (\textit{encoding region}) untuk memisahkan keduanya.

    \item \textbf{\textit{Alignment Pattern}: }
    Pola yang berfungsi mengoreksi distorsi jika kode dipindai pada permukaan melengkung atau sudut miring.
    
    \item \textbf{\textit{Timing Pattern}: }
    Garis putus-putus yang menghubungkan pola pencari untuk menentukan koordinat modul dan kepadatan simbol.
    
    \item \textbf{\textit{Quiet Zone}: }
    Area margin kosong di sekeliling simbol (minimal selebar 4 modul) yang memisahkan kode dari elemen visual di sekitarnya.
\end{enumerate}

\subsection{Koreksi Kesalahan (\textit{Error Correction})}
Salah satu keunggulan teknis QR Code yang krusial untuk implementasi \textit{e-ticket} adalah kemampuan koreksi kesalahan menggunakan algoritma Reed-Solomon. Fitur ini memungkinkan data tetap dapat dipulihkan dan dibaca meskipun sebagian area simbol rusak atau kotor \cite{tiwari2016introduction}. Tingkat koreksi kesalahan dibagi menjadi empat level sebagaimana ditampilkan pada Tabel \ref{tab:error_correction}.

\begin{table}[htbp]
    \centering
    \caption{Tingkat Koreksi Kesalahan (\textit{Error Correction Level}) pada kode QR \cite{tiwari2016introduction}}
    \label{tab:error_correction}
    % Menggunakan tabularx dengan garis vertikal (|) untuk membuat grid
    \begin{tabularx}{\textwidth}{| c | >{\raggedright\arraybackslash}X | c |}
        \hline
        \textbf{Level} & \textbf{Keterangan} & \textbf{Kemampuan Pemulihan Data} \\
        \hline
        L & \textit{Low} (Rendah) & $\approx$ 7\% \\
        \hline
        M & \textit{Medium} (Menengah) & $\approx$ 15\% \\
        \hline
        Q & \textit{Quartile} (Tinggi) & $\approx$ 25\% \\
        \hline
        H & \textit{High} (Sangat Tinggi) & $\approx$ 30\% \\
        \hline
    \end{tabularx}
\end{table}

Pemilihan level koreksi kesalahan ini menjadi pertukaran (\textit{trade-off}) antara ketahanan kode dan kapasitas data. Level M atau Q umumnya direkomendasikan untuk tiket elektronik yang berisiko mengalami kerusakan fisik (jika dicetak) atau gangguan tampilan layar \cite{tiwari2016introduction}.

\subsection{QR Code Statis vs. Dinamis}
Dalam implementasi sistem informasi, kode QR dikategorikan berdasarkan sifat data yang dikandungnya. Pemahaman terhadap perbedaan ini sangat penting dalam konteks keamanan tiket.

\begin{enumerate}[a)]
    \item Kode QR Statis: Informasi dienkodekan secara langsung dan permanen ke dalam pola matriks. Sifatnya yang \textit{fixed information} berarti data tidak dapat diubah setelah kode dibangkitkan. \textcite{yanuarafi2023perbandingan} mencatat bahwa jenis ini memiliki kelemahan keamanan karena pola visualnya yang tetap memudahkan pelaku kejahatan untuk melakukan duplikasi.
    
    \item Kode QR Dinamis (Umum): Dalam definisi pemasaran umum, kode QR dinamis menyimpan sebuah tautan pendek (\textit{short URL}) yang mengarahkan pengguna ke server tujuan. Pola QR tetap sama, namun konten di server bisa diubah. Meskipun fleksibel, pendekatan ini masih rentan terhadap penggandaan jika tautan tersebut tidak dilindungi mekanisme otentikasi tambahan.

    \item Kode QR Dinamis Berbasis Waktu (Konteks Pengerjaan Tugas Akhir): Berbeda dengan definisi pada umumnya, pengerjaan tugas akhir ini mengadopsi konsep dinamis yang muatan data (\textit{payload}) berubah secara periodik menggunakan algoritma berbasis waktu. Hal ini menyebabkan pola visual kode QR berubah total setiap interval waktu tertentu. \textcite{sung2015security} menyoroti pentingnya mekanisme kedaluwarsa (\textit{expiration}) pada kode QR untuk mencegah penggunaan ulang kode yang telah disalin. Dengan pendekatan ini, salinan tiket hasil tangkapan layar (\textit{screenshot}) akan menjadi tidak valid secara otomatis setelah durasi waktu tertentu habis.
\end{enumerate}

\section{Ancaman dan Kerentanan pada Sistem \textit{E-Ticket}}
\label{sec:keamanan_eticket}

Dalam konteks keamanan informasi, penting untuk membedakan antara ancaman (\textit{threat}) dan serangan (\textit{attack}). Ancaman merujuk pada potensi kejadian negatif yang dapat merugikan aset sistem, reputasi, atau nilai ekonomi penyedia layanan. Sementara itu, serangan adalah metode atau teknik spesifik yang dieksekusi oleh pelaku kejahatan untuk mengeksploitasi celah keamanan guna merealisasikan ancaman tersebut. Subbab ini akan menguraikan lanskap ancaman dari perspektif bisnis dan operasional, serta menganalisis vektor serangan teknis yang memungkinkan ancaman tersebut terjadi.

\subsection{Identifikasi Ancaman (\textit{Threat Landscape})}
Ancaman merepresentasikan risiko tingkat tinggi yang dihadapi oleh ekosistem pertiketan. Berdasarkan studi kasus dan literatur terkini, terdapat empat kategori ancaman utama yang menjadi fokus mitigasi:

\begin{enumerate}[a)]
    \item Praktik Percaloan (\textit{Scalping}): Calo atau makelar menurut Kamus Besar Bahasa Indonesia (KBBI) adalah orang yang menjadi perantara dan memberikan jasanya untuk menguruskan sesuatu berdasarkan upah. Ini adalah ancaman ekonomi yang terjadi ketika seseorang menjual kembali tiket yang dibelinya, namun dengan harga berkali-kali lipat dari harga normalnya yang merusak kewajaran pasar. \textcite{Pamela2023} melaporkan bahwa praktik ini sangat merugikan konsumen secara finansial dan merusak reputasi penyelenggara acara. Untuk memitigasi ancaman ini, diperlukan mekanisme validasi yang menjamin bahwa tiket yang ditampilkan adalah versi terbaru dan valid pada saat pemindaian, bukan salinan yang telah kedaluwarsa.
        
    \item Penipuan Tiket (\textit{Fraud}): Ancaman kriminal berupa penjualan tiket palsu atau tiket yang tidak valid kepada konsumen. Investigasi \textcite{diveranta2025jejak} mencatat kerugian miliaran rupiah akibat praktik ini, yang mengancam kepercayaan publik terhadap sistem penjualan tiket digital.

    \item Infiltrasi Akses Ilegal: Ancaman operasional yang terjadi ketika individu tidak berhak berhasil memasuki area acara. Hal ini tidak hanya merugikan pendapatan, tetapi juga menimbulkan risiko keamanan fisik dan ketidaknyamanan bagi pemegang tiket sah yang kursinya ditempati pihak lain \cite{Kurniawan2024}.

    \item Ancaman Kegagalan Titik Tunggal (\textit{Single Point of Failure}): Ancaman sistemik yang muncul ketika infrastruktur jaringan atau server pusat mengalami gangguan. \textcite{lever2013single} mendefinisikan \textit{Single Point of Failure} (SPoF) dalam sistem yang terintegrasi sebagai komponen kritis yang jika gagal, akan menyebabkan kegagalan operasional seluruh sistem karena terhambatnya transmisi data. Dalam konteks arsitektur server, \textcite{jafarnejad2017load} menegaskan bahwa ketergantungan pada \textit{node} pengendali terpusat (\textit{centralized}) menciptakan risiko SPoF yang tinggi; jika \textit{node} pusat tersebut mengalami gangguan atau kelebihan beban (\textit{overload}), maka seluruh layanan akan terhenti total. Hal ini sangat relevan dengan risiko kelumpuhan validasi tiket di gerbang masuk saat terjadi gangguan jaringan massal.
\end{enumerate}

\subsection{Analisis Vektor Serangan (\textit{Attack Vectors})}
Untuk mewujudkan ancaman-ancaman di atas, pelaku kejahatan menggunakan berbagai metode serangan teknis yang mengeksploitasi kelemahan pada QR Code statis. Berikut adalah analisis mengenai metode serangan tersebut:

\begin{enumerate}[a)]
    \item Serangan Penggandaan (\textit{Cloning Attack}): Serangan ini merupakan metode utama untuk melakukan penipuan tiket. Pelaku menyalin citra QR Code yang sah melalui fitur tangkapan layar (\textit{screen capture}) dan mendistribusikannya kepada korban. \textcite{sung2015security} menegaskan bahwa kerentanan utama sistem \textit{mobile} adalah kemudahan menduplikasi tampilan layar, yang disebabkan oleh sistem statis yang gagal membedakan antara citra asli di aplikasi dan citra salinan di galeri foto.
    
    \item Serangan Putar Ulang (\textit{Replay Attack}): Serangan ini mengeksploitasi validitas data tiket yang tidak memiliki batasan waktu yang ketat. Dalam skenario ini, data tiket yang sah ditangkap (disalin) dan dikirimkan ulang (\textit{replayed}) ke sistem pemindai di waktu atau lokasi berbeda. Tanpa mekanisme kedaluwarsa (\textit{expiration}), tiket yang sama dapat digunakan berulang kali untuk memasukkan banyak orang. \textcite{sung2015security} menyarankan penggunaan kedaluwarsa pada kode untuk membatalkan validitasnya setelah jangka waktu tertentu guna mematahkan serangan ini.

    \item Eksfiltrasi Data (\textit{Data Exfiltration}): Serangan ini menargetkan kerahasiaan data pengguna. \textcite{sung2015security} menjelaskan bahwa data kredensial yang disimpan tanpa enkripsi di penyimpanan lokal perangkat rentan dicuri oleh \textit{malware}. Informasi yang dicuri ini kemudian dapat digunakan oleh penyerang untuk merekonstruksi tiket valid atau melakukan pencurian identitas pengguna.
\end{enumerate}

\section{Landasan Teori Kriptografi untuk Solusi}
\label{sec:kriptografi}

Solusi keamanan yang diusulkan dalam penelitian ini, yaitu \textit{Dynamic Secure QR Code}, dibangun di atas fondasi algoritma kriptografi modern. Subbab ini akan menguraikan konsep teoretis dari teknologi kriptografi yang digunakan, meliputi kriptografi asimetris sebagai kerangka kerja utama, tanda tangan digital untuk menjamin aspek nirsangkal, serta algoritma \textit{Time-based One-Time Password} (TOTP) sebagai mekanisme pembaruan kode secara dinamis.

\subsection{Kriptografi Asimetris (\textit{Public-Key Cryptography})}
Kriptografi asimetris, atau sering disebut kriptografi kunci publik, merupakan konsep fundamental dalam keamanan informasi modern yang diperkenalkan untuk mengatasi kelemahan distribusi kunci pada kriptografi simetris. \textcite{stallings2022cryptography} menjelaskan bahwa skema ini menggunakan dua kunci berbeda yang saling berkaitan secara matematis, yaitu kunci publik dan kunci privat.

\textcite{stallings2022cryptography} menjelaskan, skema enkripsi kunci publik terdiri dari enam komponen utama yang saling berinteraksi, sebagaimana diilustrasikan pada Gambar \ref{fig:skema_pkc}. Komponen-komponen tersebut adalah:

\begin{enumerate}[a)]
    \item \textbf{\textit{Plaintext}:} Ini adalah pesan atau data asli yang dapat dibaca (\textit{readable}) yang dimasukkan ke dalam algoritma sebagai input.
    \item \textbf{Algoritma Enkripsi:} Algoritma yang melakukan berbagai transformasi matematis terhadap \textit{plaintext} untuk mengubahnya menjadi bentuk yang tidak dapat dibaca.
    \item \textbf{Kunci Publik dan Privat:} Sepasang kunci yang telah dipilih sedemikian rupa sehingga jika salah satu digunakan untuk enkripsi, maka kunci pasangannya digunakan untuk dekripsi. Transformasi pasti yang dilakukan oleh algoritma bergantung pada kunci publik atau privat yang diberikan sebagai input.
    \item \textbf{\textit{Ciphertext}:} Pesan terenkripsi atau teracak yang dihasilkan sebagai output. \textit{Ciphertext} bergantung pada \textit{plaintext} dan kunci yang digunakan. Untuk pesan yang sama, dua kunci yang berbeda akan menghasilkan dua \textit{ciphertext} yang berbeda.
    \item \textbf{Algoritma Dekripsi:} Algoritma yang menerima \textit{ciphertext} dan kunci pasangan yang cocok (kunci privat jika dienkripsi dengan publik, atau sebaliknya), lalu menghasilkan kembali \textit{plaintext} asli.
\end{enumerate}

\begin{figure}[t]
    \centering
    % --- Sub-gambar A (Posisi Atas) ---
    \begin{subfigure}[b]{\textwidth} 
        \centering
        \includegraphics[width=\textwidth]{image/encrypt_1.png} 
        \caption{Enkripsi Kunci Publik (Kerahasiaan)}
        \label{fig:enkripsi_publik}
    \end{subfigure}
    
    \vspace{1cm} 
    
    % --- Sub-gambar B (Posisi Bawah) ---
    \begin{subfigure}[b]{\textwidth}
        \centering
        \includegraphics[width=\textwidth]{image/encrypt_2.png}
        \caption{Enkripsi Kunci Privat (Autentikasi)}
        \label{fig:enkripsi_privat}
    \end{subfigure}
    
    \caption{Skema Enkripsi Kunci Publik \cite{stallings2022cryptography}}
    \label{fig:skema_pkc}
\end{figure}

Mekanisme kerja sistem ini didasarkan pada fungsi satu arah (\textit{one-way function}). Dalam skenario menjaga kerahasiaan (\textit{confidentiality}), pengirim menggunakan kunci publik penerima untuk mengenkripsi pesan, dan hanya penerima yang memiliki kunci privat pasangannya yang dapat mendekripsi pesan tersebut (Gambar \ref{fig:enkripsi_publik}). Sebaliknya, dalam skenario autentikasi, kunci privat digunakan untuk mengenkripsi (menandatangani) pesan, yang kemudian dapat diverifikasi oleh siapa saja menggunakan kunci publik (Gambar \ref{fig:enkripsi_privat}).

Pada pengerjaan tugas akhir ini, secara spesifik akan memanfaatkan algoritma \textit{Elliptic Curve Cryptography} (ECC). Berbeda dengan algoritma RSA yang mendasarkan keamanannya pada faktorisasi bilangan prima besar, ECC mendasarkan keamanannya pada masalah logaritma diskrit kurva eliptik (\textit{Elliptic Curve Discrete Logarithm Problem}). Keunggulan utama ECC adalah efisiensi sumber daya yang dijelaskan \textcite{bafandehkar2013} dalam studi perbandingannya, menunjukkan bahwa ECC mampu memberikan tingkat keamanan yang setara dengan RSA namun dengan ukuran kunci yang jauh lebih kecil. Sebagai ilustrasi, kunci ECC sebesar 160-bit menawarkan tingkat keamanan yang setara dengan kunci RSA 1024-bit. Karakteristik ini menjadikan ECC sangat ideal untuk diimplementasikan pada perangkat dengan sumber daya komputasi terbatas seperti ponsel pintar dalam sistem \textit{e-ticket}.

\subsection{Tanda Tangan Digital (\textit{Digital Signature})}
Tanda tangan digital adalah mekanisme kriptografi yang berfungsi sebagai analog digital dari tanda tangan tulisan tangan, namun dengan tingkat keamanan yang jauh lebih tinggi karena melekat secara matematis pada dokumen yang ditandatangani. Menurut \textcite{stallings2022cryptography}, tanda tangan digital memberikan tiga jaminan keamanan utama: autentikasi sumber (memastikan pengirim adalah pihak yang sah), integritas data (memastikan data tidak diubah sejak ditandatangani), dan nirsangkal (\textit{non-repudiation}) (pengirim tidak dapat menyangkal telah mengirim pesan tersebut).

Proses pembuatan tanda tangan digital melibatkan penggunaan fungsi \textit{hash} dan kunci privat pengirim. Data atau pesan (\textit{message}) terlebih dahulu diproses melalui fungsi \textit{hash} untuk menghasilkan nilai ringkasan (\textit{digest}) yang unik. Nilai \textit{hash} ini kemudian dienkripsi menggunakan kunci privat pengirim untuk membentuk tanda tangan digital. Pada sisi penerima (verifikator), proses validasi dilakukan dengan mendekripsi tanda tangan menggunakan kunci publik pengirim untuk mendapatkan nilai \textit{hash} asli, dan membandingkannya dengan nilai \textit{hash} yang dihitung ulang dari data yang diterima. Jika kedua nilai tersebut identik, maka integritas dan keaslian data terjamin. Dalam penelitian ini, algoritma yang digunakan adalah \textit{Elliptic Curve Digital Signature Algorithm} (ECDSA), yang merupakan varian dari DSA yang beroperasi pada grup kurva eliptik.

\subsection{\textit{Time-based One-Time Password} (TOTP)}
Untuk mencapai karakteristik dinamis pada sistem \textit{e-ticket}, penelitian ini mengadopsi algoritma \textit{Time-based One-Time Password} (TOTP). TOTP merupakan pengembangan dari algoritma \textit{HMAC-based One-Time Password} (HOTP) yang didefinisikan dalam standar IETF RFC 4226. HOTP membangkitkan kata sandi sekali pakai berdasarkan penghitung kejadian (\textit{event counter}) yang disinkronisasi antara klien dan server. Rumus dasar HOTP didefinisikan sebagai berikut:

\begin{equation}
    HOTP(K, C) = Truncate(HMAC\text{-}SHA\text{-}1(K, C))
\end{equation}

Keterangan:
\begin{itemize}
    \item $K$ adalah kunci rahasia bersama (\textit{shared secret key}).
    \item $C$ adalah nilai pencacah (\textit{counter}).
    \item $HMAC\text{-}SHA\text{-}1$ adalah fungsi \textit{keyed-hash message authentication code}.
\end{itemize}

Namun, HOTP memiliki kelemahan potensial berupa desinkronisasi jika tombol pembangkit ditekan berulang kali tanpa validasi ke server. Untuk mengatasi hal ini, diperkenalkan TOTP melalui standar RFC 6238. TOTP menggantikan nilai pencacah ($C$) dengan nilai waktu terkini. Algoritma ini menggunakan interval waktu (\textit{time step}) sebagai faktor pengubah, sehingga kode yang dihasilkan akan valid hanya dalam jendela waktu tertentu (misalnya 30 detik).

Perhitungan nilai langkah waktu ($T$) dalam TOTP dirumuskan sebagai berikut:

\begin{equation}
    T = \lfloor \frac{CurrentTime - T0}{X} \rfloor
\end{equation}

Keterangan:
\begin{itemize}
    \item $CurrentTime$ adalah waktu saat ini dalam detik (biasanya format \textit{Unix epoch}).
    \item $T0$ adalah waktu awal penghitungan (biasanya 0).
    \item $X$ adalah durasi langkah waktu (\textit{time step}), yang secara \textit{default} adalah 30 detik.
\end{itemize}

Dengan demikian, nilai TOTP dibangkitkan dengan memasukkan nilai $T$ ke dalam fungsi HOTP:

\begin{equation}
    TOTP = HOTP(K, T)
\end{equation}

Penggunaan TOTP menjamin bahwa \textit{payload} kode QR akan selalu berubah secara periodik mengikuti waktu server sehingga memitigasi risiko serangan penggandaan tiket (\textit{cloning attack}) dan serangan putar ulang (\textit{replay attack}) akibat penggunaan tiket hasil tangkapan layar yang telah kedaluwarsa.

\section{Mekanisme Sinkronisasi Data dan Penyimpanan Lokal}
\label{sec:batching_caching}

Untuk memitigasi risiko kegagalan jaringan dan meningkatkan kinerja sistem pada lingkungan dengan konektivitas terbatas, penelitian ini menerapkan mekanisme pengelolaan data hibrida.

\subsection{Manajemen \textit{Cache} Lokal}
Penyimpanan sementara atau \textit{caching} adalah teknik fundamental untuk efisiensi data. \textcite{tang2006benefit} menyatakan bahwa \textit{caching} data secara lokal pada node jaringan dapat secara signifikan meningkatkan efisiensi akses informasi dengan mengurangi latensi akses dan penggunaan \textit{bandwidth} jaringan. Dalam konteks validasi tiket, \textit{cache} lokal pada alat pemindai berfungsi menyimpan data kredensial tiket yang telah diverifikasi atau data kunci publik yang diperlukan sehingga memungkinkan proses validasi tetap berjalan instan (\textit{low latency}) tanpa ketergantungan penuh pada koneksi internet setiap saat.

\subsection{Sinkronisasi Asinkron (\textit{Batching})}
Selain penyimpanan lokal, efisiensi pengiriman data ke server pusat juga menjadi perhatian utama. \textcite{ramachandra2015program} menjelaskan bahwa pengiriman permintaan data secara asinkron dan terkelompok (\textit{batched}) dapat meningkatkan kinerja aplikasi secara signifikan dibandingkan pengiriman sinkron satu per satu. Teknik ini memungkinkan aplikasi untuk menumpuk log transaksi (seperti status \textit{check-in}) di sisi klien dan mengirimkannya ke server secara kolektif saat koneksi tersedia atau dalam interval waktu tertentu. Pendekatan ini mengurangi penundaan (\textit{delay}) akibat putaran jaringan (\textit{network round-trips}) yang berulang dan memastikan antrean pengunjung tidak terhambat oleh proses sinkronisasi data.
% =================================================

\section{Penelitian Terkait}
\label{sec:penelitian_terkait}

Pengembangan sistem keamanan berbasis QR Code telah menjadi subjek penelitian yang aktif dalam beberapa tahun terakhir seiring dengan meningkatnya ancaman digital. Subbab ini meninjau secara mendalam beberapa penelitian terdahulu yang relevan untuk memetakan posisi dan kontribusi penelitian ini. Tinjauan dilakukan terhadap tiga perspektif utama, yaitu: (1) mekanisme anti-pemalsuan pada media fisik, (2) analisis kerentanan pada autentikasi seluler, dan (3) studi implementasi QR Code dinamis.

\subsection{Sistem QR Code Anti-Pemalsuan Berbasis \textit{Watermarking} dan CNN \cite{alsuhibany2025innovative}}
Dalam studi ini, \textcite{alsuhibany2025innovative} mengembangkan sistem untuk memitigasi ancaman substitusi kode batang (\textit{barcode substitution fraud}) dan serangan pencetakan ulang (\textit{reprinting attack}) yang sering terjadi pada label produk dan dokumen fisik. \textcite{alsuhibany2025innovative} mengidentifikasi bahwa QR Code standar tidak memiliki fitur keamanan inheren, sehingga pelaku kejahatan dapat dengan mudah menyalin atau mengganti kode asli dengan kode palsu untuk memanipulasi informasi produk. Hal ini tidak hanya menyebabkan kerugian finansial, tetapi juga merusak kepercayaan konsumen sehingga memerlukan pengawasan manual yang lebih ketat.

Untuk mengatasi masalah tersebut, penelitian ini mengusulkan pendekatan keamanan dua lapis. Lapisan pertama adalah mekanisme \textit{tamper-proof generation} menggunakan teknik \textit{digital watermarking} pada domain spasial. Teknik ini menyisipkan pola keamanan unik (yang berbeda untuk setiap pasar) ke dalam citra QR Code menggunakan metode modifikasi \textit{Least Significant Bit} (LSB). Penulis mengklaim bahwa metode ini dipilih karena kesederhanaannya dan ketahanannya terhadap distorsi umum seperti pencetakan dan pemindaian ulang. Lapisan kedua adalah mekanisme verifikasi berbasis kecerdasan buatan (\textit{Artificial Intelligence}) menggunakan \textit{Convolutional Neural Network} (CNN). Model tersebut dilatih untuk mendeteksi perbedaan mikroskopis atau degradasi kualitas (\textit{noise}) yang membedakan antara QR Code asli dan hasil cetak ulang (\textit{reprinted}).

Meskipun metode ini terbukti efektif dalam mendeteksi pemalsuan pada media fisik, pendekatannya memiliki keterbatasan jika diterapkan pada tiket digital berbasis layar ponsel. Dalam ekosistem \textit{e-ticket}, ancaman utama adalah duplikasi melalui tangkapan layar (\textit{screenshot}) yang menghasilkan salinan digital identik secara bit-per-bit, tanpa degradasi fisik yang dapat dideteksi oleh model CNN tersebut. Oleh karena itu, solusi berbasis analisis citra statis seperti yang ditawarkan Alsuhibany perlu dilengkapi dengan mekanisme dinamis (perubahan konten) untuk mematahkan validitas salinan digital tersebut.

\subsection{Analisis Kerentanan Autentikasi Seluler Berbasis QR Code \cite{sung2015security}}
Penelitian yang dilakukan oleh \textcite{sung2015security} menyajikan analisis keamanan komprehensif terhadap sistem autentikasi yang menggunakan QR Code pada perangkat seluler. Berbeda dengan pandangan umum yang menganggap QR Code aman, penelitian ini mengungkap berbagai vektor serangan kritis, khususnya yang terjadi pada sisi klien (\textit{client-side}).

\textcite{sung2015security} mengklasifikasikan kerentanan tersebut ke dalam beberapa kategori utama. Pertama, kerentanan penggandaan (\textit{cloning}), adalah ketika QR Code mudah disalin melalui fitur tangkapan layar (\textit{screen capture}) karena sistem tidak dapat membedakan citra asli di aplikasi dengan citra salinan. Kedua, serangan putar ulang (\textit{replay attack}) yang terjadi ketika kode valid digunakan kembali di luar waktu yang diizinkan. Ketiga, eksfiltrasi data (\textit{stored data exfiltration}), yaitu risiko pencurian data kredensial yang tersimpan di memori lokal perangkat oleh aplikasi berbahaya (\textit{malware}) jika tidak dilindungi oleh enkripsi yang memadai. Selain itu, penelitian ini juga membahas ancaman lain seperti penyadapan pesan jaringan (\textit{network eavesdropping}) dan pengungkapan algoritma internal melalui teknik rekayasa balik (\textit{reverse engineering}).

Sebagai usulan mitigasi terhadap implementasi perangkat lunak, \textcite{sung2015security} mengusulkan kerangka kerja implementasi aman (\textit{secure implementation}) yang mencakup beberapa lapisan pertahanan. Rekomendasi utamanya meliputi penerapan mekanisme kedaluwarsa (\textit{expiration}) untuk mencegah serangan putar ulang, enkripsi data penyimpanan untuk mencegah eksfiltrasi, serta pengaburan kode (\textit{code obfuscation}) untuk mempersulit analisis algoritma oleh penyerang. Penelitian tugas akhir ini akan mengadopsi beberapa rekomendasi tersebut dengan cara mengimplementasikan algoritma TOTP untuk manajemen kedaluwarsa otomatis dan enkripsi asimetris pada \textit{payload} tiket guna melindungi data dari risiko eksfiltrasi dan manipulasi.

\subsection{Studi Komparasi QR Code Statis dan Dinamis \cite{yanuarafi2023perbandingan}}
Dalam konteks implementasi sistem autentikasi kehadiran, \textcite{yanuarafi2023perbandingan} melakukan studi komparatif antara penggunaan QR Code statis dan dinamis pada sistem presensi pegawai di lingkungan universitas. Penelitian ini dilatarbelakangi oleh maraknya kecurangan presensi yang terjadi akibat kelemahan sistem statis, yang terjadi akibat kode identitas yang bersifat tetap mudah disalin dan dibagikan kepada rekan kerja untuk melakukan presensi palsu (``titip absen").

Hasil penelitian menunjukkan bahwa QR Code dinamis memiliki keunggulan signifikan dalam aspek keamanan dibandingkan varian statis. Dengan mekanisme perubahan kode secara berkala, celah keamanan berupa penggunaan ulang kode (\textit{reuse}) atau penggandaan kode statis dapat diminimalisasi secara efektif. \textcite{yanuarafi2023perbandingan} menyimpulkan bahwa meskipun implementasi sistem dinamis membutuhkan sumber daya komputasi yang lebih besar, tingkat akurasi dan keamanan data yang dihasilkan jauh lebih tinggi, menjadikannya standar yang direkomendasikan untuk sistem yang membutuhkan manajemen absensi yang baik.

Meskipun penelitian ini berhasil membuktikan keunggulan konsep dinamis, fokus utamanya terletak pada fungsionalitas aplikasi presensi dan pencegahan berbagi kode secara sederhana. Penelitian tersebut belum membahas mekanisme perlindungan integritas data secara kriptografis, seperti penggunaan tanda tangan digital (\textit{digital signature}), untuk menjamin bahwa data dinamis yang dihasilkan benar-benar berasal dari otoritas yang sah dan tidak dimanipulasi oleh pihak ketiga selama proses transmisi. Celah inilah yang akan dilengkapi oleh penelitian Tugas Akhir ini melalui arsitektur \textit{Dynamic Secure QR Code}.

\subsection{Posisi Penelitian dan Kontribusi}
Berdasarkan tinjauan literatur di atas, dapat dipetakan bahwa penelitian-penelitian terdahulu umumnya berfokus pada salah satu aspek keamanan secara terpisah. Belum banyak ditemukan sistem yang mengintegrasikan mekanisme pertahanan secara holistik untuk menjawab tiga kebutuhan utama keamanan tiket, yaitu: (1) Aspek dinamis untuk mencegah serangan penggandaan, (2) aspek kerahasiaan untuk melindungi data privasi pengguna, dan (3) aspek integritas untuk menjamin keaslian penerbit tiket.

Penelitian ini bertujuan mengisi celah penelitian (\textit{research gap}) tersebut dengan mengusulkan arsitektur \textit{Dynamic Secure QR Code}. Kontribusi utama penelitian ini adalah penggabungan algoritma TOTP, enkripsi asimetris (ECC) yang efisien untuk perangkat seluler \cite{bafandehkar2013}, dan tanda tangan digital dalam satu sistem yang padu. Perbandingan posisi penelitian ini dengan penelitian terkait dapat dilihat pada Tabel \ref{tab:state_of_the_art}.

\begin{table}[H]
    \caption{Perbandingan Fitur Keamanan Penelitian Terkait dengan Penelitian yang Diusulkan}
    \label{tab:state_of_the_art}
    \centering
    \begin{tabularx}{\textwidth}{ l >{\raggedright\arraybackslash}X c c c }
        \toprule
        \textbf{Peneliti} & \textbf{Fokus} \newline \textbf{Penelitian} & \textbf{Dinamis} & \textbf{Rahasia} & \textbf{Integritas} \\
        \midrule
        \textcite{sung2015security} & Analisis kerentanan autentikasi \textit{mobile} & Saran & Saran & - \\
        \addlinespace
        \textcite{alsuhibany2025innovative} & \textit{Watermarking} digital pada media cetak & Tidak & Tidak & Ya \\
        \addlinespace
        \textcite{yanuarafi2023perbandingan} & Perbandingan presensi statis vs dinamis & Ya & Tidak & - \\
        \addlinespace
        \textbf{Penelitian Ini} & \textbf{Sistem \textit{E-Ticket} (TOTP + Enkripsi + TTD)} & \textbf{Ya} & \textbf{Ya} & \textbf{Ya} \\
        \bottomrule
    \end{tabularx}
\end{table}