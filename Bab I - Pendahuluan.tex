% ==========================================
% BAB I PENDAHULUAN
% ==========================================
\chapter{PENDAHULUAN}
\label{chap:pendahuluan}
% --- Latar Belakang ---
\section{Latar Belakang}

Berdasarkan Kamus Besar Bahasa Indonesia (KBBI), Tiket atau karcis adalah surat kecil (carik kertas khusus) sebagai tanda telah membayar ongkos dan sebagainya (untuk naik bus, menonton bioskop, dan sebagainya). Tiket merupakan sebuah dokumen yang berfungsi sebagai bukti hak akses atau tanda pembayaran yang sah untuk menggunakan suatu layanan atau memasuki suatu area tertentu. Secara historis, tiket konvensional dalam bentuk fisik telah menjadi bagian tak terpisahkan dari berbagai sektor, mulai dari transportasi hingga hiburan.  Namun, seiring dengan pesatnya perkembangan teknologi informasi, terjadi pergeseran paradigma menuju digitalisasi tiket menjadi tiket elektronik (\textit{e-ticket}). Inovasi layanan ini sangat erat kaitannya dengan adopsi sistem teknis berbasis komputer yang memungkinkan peningkatan efisiensi dan efektivitas operasional \cite{lubeck2012electronic}. Pergeseran paradigma tersebut didorong oleh kebutuhan untuk meningkatkan manajemen informasi yang sebelumnya sulit dilakukan dengan sistem manual atau kartu magnetik \cite{lubeck2012electronic}. 

Adopsi \textit{e-ticket} mulai marak pada awal tahun 2000-an, yang dipelopori oleh industri penerbangan di tahun 1990-an, dan kini telah diadopsi secara masif di berbagai sektor. \textit{E-ticket} menawarkan berbagai keunggulan signifikan dibandingkan tiket konvensional yang rentan terhadap inefisiensi. \textcite{lubeck2012electronic} menyoroti bahwa sistem konvensional seringkali terkendala oleh lemahnya kontrol operasional yang menyebabkan maraknya perdagangan tiket ilegal serta penyalahgunaan manfaat tiket khusus (seperti tiket pelajar) karena sulitnya identifikasi pengguna. Dari sisi pengguna, \textit{e-ticket} memberikan kemudahan distribusi dan akses, menghilangkan risiko kehilangan tiket fisik, serta membantu menghindari antrean panjang. Selain itu, sistem ini juga lebih efisien dari segi biaya operasional dan memiliki dampak lingkungan yang lebih rendah karena mengurangi penggunaan kertas \cite{chen2007passenger}.

Untuk merealisasikan berbagai keunggulan \textit{e-ticket} tersebut, diperlukan medium representasi data yang efisien dan kompatibel dengan perangkat pengguna. Di antara berbagai alternatif teknologi, \textit{Quick Response Code} (QR Code) muncul sebagai solusi dominan yang diadopsi secara luas dalam implementasi \textit{e-ticket}. QR Code adalah jenis kode batang (\textit{barcode}) matriks atau kode dua dimensi yang dapat menyimpan informasi digital \cite{shin2012psychology}. Tidak seperti \textit{barcode} satu dimensi, QR Code mengenkode data secara horizontal dan vertikal, menawarkan kepadatan informasi yang lebih tinggi dan kecepatan pembacaan yang lebih cepat \cite{alsuhibany2025innovative}. \textcite{tiwari2016introduction} menjelaskan bahwa tingkat penerimaan QR Code yang tinggi secara global berbanding lurus dengan pertumbuhan pengguna ponsel pintar, yang memungkinkan teknologi ini menjangkau konsumen secara luas dan cepat. Ubikuitas perangkat pemindai yang terintegrasi dalam ponsel pintar, menjadikan QR Code pilihan yang praktis dan efisien untuk diterapkan sebagai medium \textit{e-ticket}. Kepopuleran dan kemudahan akses tersebut mendorong adopsi luas QR Code pada gerbang transportasi maupun acara hiburan. Akan tetapi, di balik kenyamanan tersebut, model \textit{e-ticket} konvensional yang mengandalkan QR Code dalam bentuk statis, secara inheren mewarisi celah keamanan yang serius.

Sistem \textit{e-ticket} pada umumnya mengadopsi model QR Code statis. Pada model ini, data tiket seperti identitas pengguna atau tautan validasi, dienkode secara langsung ke dalam pola matriks citra. Karakteristik fundamental dari QR Code statis adalah informasi yang tersimpan di dalamnya bersifat tetap (\textit{fixed information}) \cite{yanuarafi2023perbandingan}; artinya, setelah kode dibangkitkan (\textit{generated}), pola visualnya tidak akan berubah dan terus valid sepanjang masa berlaku tiket. Proses validasi bergantung sepenuhnya pada pemindaian di pintu masuk, yaitu saat alat pemindai menerjemahkan kembali pola matriks menjadi data identitas untuk dicocokkan dengan basis data. Meskipun arsitektur ini menawarkan kemudahan implementasi, menurut \textcite{yanuarafi2023perbandingan}, penggunaan QR Code statis memiliki kelemahan signifikan dalam aspek keamanan. Sifatnya yang permanen membuat sistem ini rentan terhadap penyalahgunaan, seperti duplikasi ilegal dan pemalsuan, yang pada akhirnya mengancam integritas ekosistem \textit{e-ticket} secara keseluruhan.

% --- [PARAGRAF 5: DAMPAK EKSPLOITASI (REVISI LOGIKA MASALAH)] ---
Kelemahan mendasar dari arsitektur statis adalah sifatnya yang ``sekali terbit, berlaku selamanya'' tanpa mekanisme pembaruan autentikasi. Celah tersebut dieksploitasi secara luas melalui serangan penggandaan (\textit{cloning}) dan serangan putar ulang (\textit{replay attack}), di mana citra QR Code yang sah disalin dan digunakan kembali oleh pihak tidak berwenang \cite{lee2016security}. Dampak dari kerentanan ini sangat masif, terutama pada acara dengan permintaan tinggi. Sebagai contoh kasus, pada konser Coldplay di Jakarta tahun 2023, ditemukan fenomena di mana satu tiket asli digandakan dan dijual kepada puluhan orang berbeda. Hal ini mengakibatkan kekacauan di pintu masuk ketika pemegang tiket sah gagal melakukan pemindaian karena tiketnya telah digunakan lebih dulu oleh pembeli tiket duplikat \cite{Pamela2023}. Lebih jauh lagi, kemudahan dalam memalsukan tampilan visual QR Code statis turut menyuburkan praktik penipuan tiket. Pusat Pelaporan dan Analisis Transaksi Keuangan (PPATK) mencatat indikasi transaksi mencurigakan terkait penipuan tiket senilai miliaran rupiah pada tahun 2024. Tingginya angka penipuan ini sebagian besar didorong oleh ketidakmampuan sistem konvensional dalam menyediakan mekanisme verifikasi mandiri bagi pembeli, sehingga mereka mudah diperdaya oleh tiket palsu yang secara visual tampak identik dengan tiket asli.

% --- [PARAGRAF 6: KONSEP SOLUSI (REVISI MENJADI LEBIH KONSEPTUAL)] ---
Rentetan permasalahan tersebut---mulai dari duplikasi tiket (\textit{double spending}) hingga pemalsuan dokumen---membuktikan bahwa sistem verifikasi yang hanya mengandalkan QR Code statis tidak lagi memadai. Diperlukan sebuah pendekatan komprehensif untuk menjamin integritas transaksi dan data. Berdasarkan analisis masalah tersebut, sebuah sistem \textit{e-ticket} yang ideal harus memiliki tiga karakteristik pertahanan utama. Pertama, tiket harus bersifat dinamis (\textit{dynamic}) menggunakan mekanisme pembangkitan kode yang berubah secara berkala berbasis waktu, sehingga tangkapan layar menjadi tidak valid setelah durasi tertentu. Kedua, tiket harus menjamin kerahasiaan (\textit{confidentiality}) melalui enkripsi muatan data (\textit{payload}) untuk melindungi privasi pengguna dari pembacaan data sembarangan. Ketiga, tiket harus bersifat nirsangkal (\textit{secure}) menggunakan mekanisme tanda tangan digital (\textit{digital signature}) untuk memastikan tiket diterbitkan oleh otoritas yang sah dan tidak dimodifikasi.

% --- [PARAGRAF 7: TUJUAN & URGENSI (TETAP/MINOR ADJUSTMENT)] ---
Oleh karena itu, penelitian ini bertujuan untuk merancang dan mengimplementasikan sistem \textit{e-ticket} yang mengusung konsep \textit{Dynamic Secure QR Code}. Urgensi penelitian ini difokuskan pada sektor hiburan dan olahraga skala besar, mengingat sektor ini memiliki risiko kerugian tertinggi akibat manipulasi tiket. Melalui implementasi sistem ini, diharapkan tercipta ekosistem pertiketan yang lebih sehat yang memberikan manfaat ganda: konsumen mendapatkan jaminan perlindungan hak akses dan data pribadi, sementara penyelenggara acara dapat memitigasi kebocoran pendapatan (\textit{revenue leakage}) akibat tiket palsu. Penelitian ini akan berfokus pada pengembangan prototipe sistem yang mampu membangkitkan dan memvalidasi tiket dengan arsitektur keamanan berlapis tersebut.

% --- Rumusan Masalah ---
\section{Rumusan Masalah}
Berdasarkan latar belakang yang telah diuraikan, teridentifikasi adanya celah keamanan pada sistem \textit{e-ticketing} yang umum digunakan saat ini, terutama yang berbasis QR Code statis. Celah ini menimbulkan berbagai permasalahan seperti duplikasi, pemalsuan, dan percaloan tiket digital. Oleh karena itu, rumusan masalah dalam penelitian ini adalah sebagai berikut:
\begin{enumerate}
\item	Bagaimana merancang dan mengimplementasikan sebuah sistem \textit{e-ticket} yang memanfaatkan konsep \textit{Dynamic Secure QR Code} untuk mengatasi celah keamanan pada sistem pertiketan konvensional?
\item	Bagaimana alur proses untuk menghasilkan QR Code yang bersifat dinamis (berubah secara periodik) dan aman (terenkripsi dan memiliki tanda tangan digital) serta proses validasinya?
\item	Bagaimana unjuk kerja sistem yang diusulkan dalam memitigasi risiko duplikasi tiket melalui tangkapan layar (\textit{screenshot}) dan pemalsuan tiket dibandingkan dengan sistem QR Code statis?
\end{enumerate}

% --- Tujuan ---
\section{Tujuan}
Penelitian ini bertujuan untuk merancang dan mengimplementasikan sebuah prototipe (\textit{proof-of-concept}) sistem \textit{e-ticket} yang mampu menghasilkan dan memvalidasi \textit{Dynamic Secure QR Code}. Secara spesifik, penelitian ini akan mendefinisikan dan membangun alur proses teknis yang terperinci untuk generasi dan validasi (\textit{e-ticket}), yang mencakup proses enkripsi data, implementasi token berbasis waktu, dan penggunaan Tanda Tangan Digital. Pada akhirnya, penelitian ini akan melakukan serangkaian pengujian dan analisis terhadap unjuk kerja prototipe untuk membuktikan secara empiris efektivitasnya dalam memitigasi risiko keamanan utama, khususnya dalam mencegah duplikasi tiket melalui serangan tangkapan layar (\textit{screenshot}) dan upaya pemalsuan tiket.

% --- Batasan Masalah ---
\section{Batasan Masalah}
Agar penelitian ini dapat lebih fokus dan mendalam, maka ruang lingkup permasalahan dibatasi sebagai berikut:
\begin{enumerate}
\item	Penelitian ini berfokus pada perancangan dan implementasi modul inti, yaitu proses generasi (\textit{generation}) dan validasi (\textit{validation}) \textit{Dynamic Secure QR Code}.
\item	Penelitian ini tidak akan membangun sistem \textit{e-commerce} atau \textit{marketplace} penjualan tiket secara lengkap. Fitur seperti manajemen pengguna, sistem pembayaran, dan manajemen acara tidak menjadi fokus utama.
\item	Sistem yang dibangun berupa prototipe (\textit{proof-of-concept}) untuk menunjukkan kelayakan konsep, bukan aplikasi skala produksi yang siap dirilis secara komersial.
\item   Implementasi teknis prototipe akan menggunakan bahasa pemrograman Python beserta \textit{library} kriptografi dan QR Code yang relevan.
\item   Penelitian tidak mencakup perancangan perangkat keras \textit{(hardware)} pemindai khusus. Proses pemindaian diasumsikan dapat dilakukan menggunakan kamera pada perangkat ponsel pintar.
\end{enumerate}

% --- Metodologi Pengerjaan TA ---
\section{Metodologi}
Tuliskan semua tahapan yang akan dilalui selama pelaksanaan tugas akhir. Tahapan ini spesifik untuk menyelesaikan persoalan tugas akhir. Khusus untuk penyusunan proposal ini, jelaskan secara detail:
\begin{enumerate}
\item Tahapan Investigasi dan Perumusan Masalah
    
Tahap awal ini berfokus pada pengumpulan fakta untuk mengidentifikasi dan memvalidasi masalah yang ada di latar belakang. Proses ini dilakukan melalui:
\begin{enumerate}
\item \textit{Observasi Fenomena:} Mengamati tren transformasi digital dalam sistem pertiketan di Indonesia, yang menunjukkan pergeseran masif dari tiket fisik ke \textit{e-ticket}, terutama yang menggunakan teknologi QR Code.
\item \textit{Analisis Data Sekunder:} Melakukan pengumpulan dan analisis data dari sumber-sumber kredibel di media massa \textit{online}. Proses ini melibatkan pencarian berita terkait insiden kegagalan sistem tiket pada acara-acara berskala besar.
\item \textit{Sintesis Masalah:} Menganalisis dan mensintesiskan temuan dari berbagai insiden tersebut untuk merumuskan akar permasalahan utama, yaitu kerentanan QR Code statis terhadap duplikasi ilegal, pemalsuan, dan kelemahan proses validasi.

\end{enumerate}


\item Tahapan Studi Literatur dan Pencarian Solusi    

Tahap ini bertujuan untuk membangun landasan teoretis yang kuat dan mengidentifikasi pendekatan solusi yang paling tepat. Langkah-langkah yang dilakukan adalah sebagai berikut:
\begin{enumerate}
\item \textit{Identifikasi Landasan Teori:} Menentukan konsep dan teori fundamental yang perlu digali, yang mencakup: teknologi QR Code, Kriptografi Asimetris, Tanda Tangan Digital, dan \textit{Time-based One-Time Password} (TOTP).
\item \textit{Pencarian Literatur:} Melakukan pencarian sistematis pada basis data akademik untuk menemukan penelitian-penelitian relevan yang telah dilakukan oleh orang lain dengan kata kunci seperti: \textit{"secure QR code ticketing"}, \textit{"e-ticket security"}, dan \textit{"QR code replay attack"}.
\item \textit{Seleksi dan Sintesis Literatur:} Melakukan seleksi dan sintesis terhadap literatur yang ditemukan untuk membangun argumen bahwa kombinasi Tanda Tangan Digital dan TOTP merupakan pendekatan solusi yang solid dan inovatif. Hasil dari tahap ini akan didokumentasikan pada Bab II.
\end{enumerate}


\item Tahapan Perancangan Sistem    

Merancang arsitektur sistem, alur proses generasi dan validasi QR Code, struktur data \textit{payload}, dan antarmuka antar modul.

\item Tahapan Implementasi Prototipe
    
Mengembangkan perangkat lunak prototipe menggunakan bahasa pemrograman Python dan \textit{library} terkait untuk merealisasikan rancangan sistem.

\item Tahapan Pengujian Sistem
    
Melakukan pengujian fungsional dan keamanan dengan skenario spesifik untuk memverifikasi kemampuan sistem dalam menolak tiket hasil \textit{screenshot} dan tiket palsu.

\item Tahapan Analisis dan Penarikan Kesimpulan
    
Menganalisis data hasil pengujian untuk mengevaluasi kinerja dan efektivitas prototipe, yang kemudian akan menjadi dasar untuk penarikan kesimpulan penelitian.
    
\item Tahapan Penyusunan Laporan

Mendokumentasikan seluruh proses dan hasil penelitian ke dalam format laporan Tugas Akhir yang komprehensif.

\end{enumerate}