% ==========================================
% BAB I PENDAHULUAN
% ==========================================
\chapter{PENDAHULUAN}
\label{chap:pendahuluan}
% --- Latar Belakang ---
\section{Latar Belakang}

Berdasarkan Kamus Besar Bahasa Indonesia (KBBI), Tiket atau karcis adalah surat kecil (carik kertas khusus) sebagai tanda telah membayar ongkos dan sebagainya (untuk naik bus, menonton bioskop, dan sebagainya). Tiket merupakan sebuah dokumen yang berfungsi sebagai bukti hak akses atau tanda pembayaran yang sah untuk menggunakan suatu layanan atau memasuki suatu area tertentu. Secara historis, tiket konvensional dalam bentuk fisik telah menjadi bagian tak terpisahkan dari berbagai sektor, mulai dari transportasi hingga hiburan.  Namun, seiring dengan pesatnya perkembangan teknologi informasi, terjadi pergeseran paradigma menuju digitalisasi tiket menjadi tiket elektronik (\textit{e-ticket}). Inovasi layanan ini sangat erat kaitannya dengan adopsi sistem teknis berbasis komputer yang memungkinkan peningkatan efisiensi dan efektivitas operasional \cite{lubeck2012electronic}. Pergeseran paradigma tersebut didorong oleh kebutuhan untuk meningkatkan manajemen informasi yang sebelumnya sulit dilakukan dengan sistem manual atau kartu magnetik \cite{lubeck2012electronic}. 

Adopsi \textit{e-ticket} mulai marak pada awal tahun 2000-an, yang dipelopori oleh industri penerbangan di tahun 1990-an, dan kini telah diadopsi secara masif di berbagai sektor. \textit{E-ticket} menawarkan berbagai keunggulan signifikan dibandingkan tiket konvensional yang rentan terhadap inefisiensi. \textcite{lubeck2012electronic} menyoroti bahwa sistem konvensional seringkali terkendala oleh lemahnya kontrol operasional yang menyebabkan maraknya perdagangan tiket ilegal serta penyalahgunaan manfaat tiket khusus (seperti tiket pelajar) karena sulitnya identifikasi pengguna. Dari sisi pengguna, \textit{e-ticket} memberikan kemudahan distribusi dan akses, menghilangkan risiko kehilangan tiket fisik, serta membantu menghindari antrean panjang. Selain itu, sistem ini juga lebih efisien dari segi biaya operasional karena mengurangi penggunaan kertas dan menghindari komisi yang dibayarkan kepada sistem distribusi dan agen.\cite{chen2007passenger}.

Untuk merealisasikan berbagai keunggulan \textit{e-ticket} tersebut, diperlukan medium representasi data yang efisien dan kompatibel dengan perangkat pengguna. Di antara berbagai alternatif teknologi, \textit{Quick Response Code} (QR Code) muncul sebagai solusi dominan yang diadopsi secara luas dalam implementasi \textit{e-ticket}. QR Code adalah jenis kode batang (\textit{barcode}) matriks atau kode dua dimensi yang dapat menyimpan informasi digital \cite{shin2012psychology}. Tidak seperti \textit{barcode} satu dimensi, QR Code mengenkode data secara horizontal dan vertikal, menawarkan kepadatan informasi yang lebih tinggi dan kecepatan pembacaan yang lebih cepat \cite{alsuhibany2025innovative}. \textcite{tiwari2016introduction} menjelaskan bahwa tingkat penerimaan QR Code yang tinggi secara global berbanding lurus dengan pertumbuhan pengguna ponsel pintar, yang memungkinkan teknologi ini menjangkau konsumen secara luas dan cepat. Ubikuitas perangkat pemindai yang terintegrasi dalam ponsel pintar, menjadikan QR Code pilihan yang praktis dan efisien untuk diterapkan sebagai medium \textit{e-ticket}. Kepopuleran dan kemudahan akses tersebut mendorong adopsi luas QR Code pada gerbang transportasi maupun acara hiburan. Akan tetapi, di balik kenyamanan tersebut, model \textit{e-ticket} konvensional yang mengandalkan QR Code dalam bentuk statis, secara inheren mewarisi celah keamanan yang serius.

Sistem \textit{e-ticket} pada umumnya mengadopsi model kode QR statis. Pada model ini, data tiket seperti identitas pengguna atau tautan validasi, dienkode secara langsung ke dalam pola matriks citra. Karakteristik fundamental dari kode QR statis adalah informasi yang tersimpan di dalamnya bersifat tetap (\textit{fixed information}) \cite{yanuarafi2023perbandingan}; artinya, setelah kode dibangkitkan (\textit{generated}), pola visualnya tidak akan berubah dan terus valid sepanjang masa berlaku tiket. Proses validasi bergantung sepenuhnya pada pemindaian di pintu masuk, yaitu saat alat pemindai menerjemahkan kembali pola matriks menjadi data identitas untuk dicocokkan dengan basis data. Meskipun arsitektur ini menawarkan kemudahan implementasi, menurut \textcite{yanuarafi2023perbandingan}, penggunaan kode QR statis memiliki kelemahan signifikan dalam aspek keamanan. Sifatnya yang permanen membuat sistem ini rentan terhadap penyalahgunaan, seperti duplikasi ilegal dan pemalsuan, yang pada akhirnya mengancam integritas ekosistem \textit{e-ticket} secara keseluruhan.

Kelemahan mendasar dari arsitektur statis adalah sifatnya yang ``sekali terbit, berlaku selamanya'' tanpa mekanisme pembaruan autentikasi. Celah tersebut dieksploitasi secara luas melalui serangan penggandaan (\textit{cloning}) dan serangan putar ulang (\textit{replay attack}). \textcite{sung2015security} dalam analisis keamanannya menegaskan bahwa kode QR sangat mudah diduplikasi melalui fitur tangkapan layar (\textit{screen capture}) pada perangkat seluler, yang kemudian dapat ditransfer ke pihak lain tanpa bisa dicegah oleh sistem konvensional. Dampak dari kerentanan ini menciptakan efek domino kerusakan pada ekosistem pertiketan. 

Pertama, pada aspek validasi di lapangan, insiden konser Coldplay di Jakarta tahun 2023 memperlihatkan kekacauan di pintu masuk ketika banyak pemegang tiket sah gagal mendapatkan akses karena tiket mereka telah digandakan dan digunakan lebih dulu oleh pihak lain. Berdasarkan analisis hukum, modus ini terjadi karena pelaku mempelajari desain visual tiket statis lalu menggandakannya untuk dijual ke banyak korban \cite{berma2023analisis}. Kedua, lemahnya sistem keamanan turut menyuburkan praktik percaloan (\textit{scalping}), yaitu dengan menjual kembali tiket yang telah dibeli secara legal, dengan harga berkali-kali lipat dari harga resmi sehingga merusak kewajaran pasar \cite{Pamela2023}. Ketiga, kegagalan kontrol akses berlanjut hingga ke dalam arena, seperti pada salah satu pertandingan Timnas Indonesia di GBK. Pada kasus tersebut, penonton tanpa hak akses valid berhasil masuk dan menduduki kursi pemegang tiket sah, memicu konflik fisik dan ketidaknyamanan \cite{Kurniawan2024}. Terakhir, dari sisi kerugian materiil, investigasi Kompas mengungkapkan data Pusat Pelaporan dan Analisis Transaksi Keuangan (PPATK) yang mencatat 182 kasus transaksi mencurigakan terkait penipuan tiket konser pada tahun 2024 dengan total nilai Rp 2,3 miliar \cite{diveranta2025jejak}. Rangkaian kasus ini menegaskan bahwa sistem konvensional saat ini gagal memberikan perlindungan menyeluruh, baik dari sisi keamanan akses, keadilan harga, maupun perlindungan hak konsumen.

Kompleksitas permasalahan tersebut mulai dari kekacauan validasi fisik, inflasi harga akibat percaloan, hingga kerugian materiil akibat penipuan, membuktikan bahwa sistem verifikasi yang hanya mengandalkan kode QR statis tidak lagi memadai. Diperlukan sebuah pendekatan komprehensif untuk menjamin integritas transaksi dan data. Berdasarkan analisis masalah tersebut, sebuah \textit{e-ticket} yang ideal harus memiliki tiga karakteristik pertahanan utama. Pertama, tiket harus bersifat dinamis (\textit{dynamic}) menggunakan mekanisme pembangkitan kode QR yang berubah secara berkala berbasis waktu sehingga tangkapan layar menjadi tidak valid setelah durasi tertentu \cite{sung2015security}. Kedua, tiket harus mengutamakan pelindungan privasi (\textit{privacy preservation}) melalui penerapan prinsip minimalisasi data (\textit{data minimization}). Hal ini dicapai dengan membatasi muatan data (\textit{payload}) pada kode QR hanya untuk atribut teknis non-sensitif guna mencegah risiko eksfiltrasi data pribadi pengguna (\textit{PII leak}) dari penyimpanan lokal. Ketiga, tiket harus bersifat aman (\textit{secure}) menggunakan mekanisme tanda tangan digital (\textit{digital signature}) yang menjamin aspek nirsangkal (\textit{non-repudiation}), untuk memastikan tiket diterbitkan oleh otoritas yang sah dan tidak dimodifikasi

Namun, pengamanan data tiket hanyalah satu sisi dari solusi. Tantangan lain yang tak kalah penting dalam penyelenggaraan acara berskala besar adalah risiko operasional akibat ketergantungan penuh pada konektivitas internet, atau dikenal sebagai Titik Kegagalan Tunggal (\textit{Single Point of Failure}). Arsitektur sistem konvensional yang mewajibkan setiap pemindaian tiket terhubung langsung ke server pusat sangat rentan lumpuh saat terjadi lonjakan beban trafik atau gangguan jaringan. Kerentanan ini terbukti dari insiden kendala teknis pada \textit{platform} Ticketmaster saat penjualan tiket konser Taylor Swift, yang menunjukkan bahwa server pusat memiliki batas toleransi beban yang nyata dan dapat lumpuh seketika akibat lonjakan permintaan \cite{cockroach2023taylor}. Risiko serupa juga mengintai infrastruktur awan (\textit{cloud}). Laporan insiden Amazon Web Services (AWS) pada tahun 2017 memperlihatkan bagaimana kesalahan teknis pada satu layanan inti dapat menyebabkan kegagalan berantai pada sistem lain yang bergantung padanya \cite{aws2017summary}. Jika server validasi tiket mengalami gangguan serupa saat acara berlangsung, alat pemindai di lokasi akan kehilangan fungsinya dan memicu kemacetan fatal di gerbang masuk.

Oleh karena itu, untuk menjamin keberlangsungan proses validasi di tengah ketidakpastian kondisi jaringan, diperlukan mekanisme validasi mandiri (\textit{offline validation}) pada perangkat pemindai. Agar perangkat dapat memvalidasi tiket secara mandiri tanpa menghubungi server, diperlukan strategi manajemen \textit{cache} lokal yang menyimpan kredensial validasi (seperti kunci publik dan daftar tiket) secara aman di sisi perangkat. Selanjutnya, karena validasi dilakukan secara lokal, tantangan berikutnya adalah bagaimana menyinkronkan status penggunaan tiket kembali ke server pusat tanpa membebani jaringan secara \textit{real-time}. Untuk menjawab hal ini, diterapkan mekanisme sinkronisasi data asinkron atau \textit{batching}, yaitu data log validasi dikirimkan secara berkala atau saat koneksi stabil sehingga integritas data terjaga tanpa mengorbankan kecepatan validasi di lapangan.

Dengan demikian, penelitian ini mengusulkan pengembangan sistem \textit{Dynamic Secure QR Code} yang tidak hanya fokus pada aspek keamanan anti-percaloan melalui algoritma token dinamis, tetapi juga mengintegrasikan mekanisme validasi hibrida (kombinasi \textit{online} dan \textit{offline}). Melalui pemanfaatan \textit{cache} lokal dan sinkronisasi \textit{batching}, sistem ini diharapkan mampu menghadirkan solusi pertiketan yang aman dari pemalsuan sekaligus tangguh (\textit{resilient}) terhadap gangguan infrastruktur jaringan.

% --- Rumusan Masalah ---
\section{Rumusan Masalah}
Berdasarkan latar belakang yang telah diuraikan, teridentifikasi adanya kelemahan fundamental pada arsitektur \textit{e-ticket} berbasis Kode QR statis yang rentan terhadap berbagai eksploitasi keamanan. Oleh karena itu, rumusan masalah dalam penelitian ini adalah sebagai berikut:
\begin{enumerate}
    \item Bagaimana merancang arsitektur sistem \textit{e-ticket} berbasis token dinamis (\textit{Dynamic Secure QR Code}) yang mampu memitigasi serangan penggandaan tiket sekaligus menjamin privasi pengguna melalui pendekatan minimalisasi data?
    \item Bagaimana mekanisme validasi tiket yang dapat beroperasi secara mandiri (\textit{offline}) pada perangkat pemindai untuk mencegah kegagalan sistem (\textit{Single Point of Failure}) saat terjadi gangguan konektivitas jaringan?
    \item Bagaimana menjaga konsistensi status penggunaan tiket antara penyimpanan lokal di perangkat pemindai dan server pusat menggunakan mekanisme sinkronisasi asinkron (\textit{batching}) tanpa mengganggu kinerja validasi di lapangan?
\end{enumerate}

% --- Tujuan ---
\section{Tujuan Penelitian}
Mengacu pada rumusan masalah di atas, tujuan utama dari penelitian ini adalah:
\begin{enumerate}
    \item Merancang dan mengimplementasikan arsitektur sistem \textit{e-ticket} berbasis token dinamis (\textit{Dynamic Secure QR Code}) yang memanfaatkan algoritma pembangkitan kode berbasis waktu dan tanda tangan digital untuk menjamin keaslian tiket serta melindungi data pribadi pengguna.
    \item Mengembangkan mekanisme validasi mandiri (\textit{offline validation}) pada perangkat pemindai dengan memanfaatkan manajemen penyimpanan lokal (\textit{local cache}), teknik derivasi kunci (\textit{Key Derivation}), dan kriptografi kunci publik sehingga sistem tetap andal tanpa ketergantungan koneksi server terus-menerus.
    \item Menerapkan mekanisme sinkronisasi data asinkron (\textit{batching}) untuk menjamin integritas dan konsistensi data status tiket antara perangkat pemindai dan server pusat secara efisien tanpa membebani kinerja operasional di lapangan.
\end{enumerate}

% --- Batasan Masalah ---
\section{Batasan Masalah}
Agar pengerjaan tugas akhir dapat lebih terarah dan tidak melenceng dari tujuan utamanya, ruang lingkup permasalahan dibatasi sebagai berikut:
\begin{enumerate}
    \item Penelitian ini berfokus pada perancangan dan implementasi modul inti keamanan, yaitu proses pembangkitan (\textit{generation}) dan validasi (\textit{validation}) \textit{Dynamic Secure QR Code}, tanpa membahas aspek antarmuka pengguna (UI/UX) secara mendalam.
    \item Penelitian ini tidak mencakup pengembangan fitur sistem penjualan tiket yang kompleks (seperti manajemen akun pengguna, integrasi gerbang pembayaran, atau manajemen acara), melainkan fokus pada siklus hidup tiket mulai dari penerbitan hingga validasi.
    \item Luaran sistem yang dibangun berupa prototipe (\textit{proof-of-concept}) yang bertujuan untuk mendemonstrasikan kelayakan logika keamanan dan mekanisme \textit{offline}, bukan sebagai aplikasi skala produksi yang siap rilis komersial.
    \item Sistem menerapkan prinsip \textit{Data Minimization}, yaitu muatan data (\textit{payload}) pada Kode QR tidak dienkripsi, melainkan hanya berisi informasi non-sensitif (seperti ID referensi). Keamanan data sensitif pengguna diasumsikan terjamin pada basis data server pusat.
    \item Implementasi teknis prototipe akan dikembangkan menggunakan teknologi perangkat lunak yang relevan dan mendukung pustaka (\textit{library}) kriptografi standar, tanpa terikat pada satu bahasa pemrograman spesifik.
    \item Penelitian tidak mencakup perancangan perangkat keras (\textit{hardware}) pemindai khusus. Proses pemindaian dan validasi diasumsikan dilakukan menggunakan perangkat lunak pada ponsel pintar (\textit{smartphone}) yang memanfaatkan kamera bawaan.
    \item Perangkat pemindai diasumsikan sebagai perangkat terpercaya (\textit{trusted device}) yang memiliki mekanisme keamanan fisik memadai. Risiko pencurian fisik perangkat pemindai dan teknik \textit{reverse engineering} untuk mengekstrak kunci rahasia utama (\textit{Master Secret}) dianggap sebagai risiko operasional di luar lingkup sistem yang dirancang.
    \item Implementasi dan pengujian sistem diasumsikan menggunakan satu perangkat pemindai (\textit{single gate scanner}) pada satu waktu. Isu sinkronisasi data secara \textit{real-time} (\textit{peer-to-peer}) antar-banyak pemindai dalam kondisi \textit{offline} berada di luar lingkup penelitian ini.
\end{enumerate}

% --- Metodologi ---
\section{Metodologi}
Pengerjaan tugas akhir ini menerapkan kerangka kerja \textit{Software Development Life Cycle} (SDLC) dengan pendekatan model \textit{Waterfall} sebagai metodologi. Model ini dipilih karena pengerjaan tugas akhir yang memiliki kebutuhan sistem (\textit{requirements}) yang didefinisikan secara jelas di tahap awal, yaitu berfokus pada aspek keamanan QR Code, serta membutuhkan alur pengerjaan yang terstruktur. Tahapan pengembangan sistem dalam pengerjaan tugas akhir mengacu pada standar rekayasa perangkat lunak menurut \textcite{sommerville2016software}, yang secara visual dapat dilihat pada Gambar \ref{fig:metodologi_waterfall}.

\begin{figure}[h]
    \centering
    \captionsetup{justification=centering}
    	\includegraphics[width=0.9\textwidth]{image/waterfall.png}
    \caption{Alur Metodologi Penelitian Model Waterfall}
    \label{fig:metodologi_waterfall}
\end{figure}

Rincian tahapan yang akan dilalui selama pelaksanaan tugas akhir adalah sebagai berikut:

\begin{enumerate}
    \item \textbf{Analisis Kebutuhan (\textit{Requirements Analysis})} \\
    Tahapan ini merupakan langkah fundamental untuk mengumpulkan fakta empiris dan merumuskan spesifikasi kebutuhan sistem. Proses investigasi dilakukan dengan mengobservasi fenomena kegagalan sistem \textit{e-ticket} (\textit{system crash}) pada acara berskala besar, serta mengumpulkan data sekunder dari sumber kredibel, seperti laporan PPATK dan pemberitaan media massa terkait modus kejahatan tiket. Selain itu, dilakukan studi literatur terhadap mekanisme \textit{offline-first}, sinkronisasi data (\textit{batching}), serta standar teknis kriptografi seperti \textit{Time-based One-Time Password} (TOTP) dan Tanda Tangan Digital untuk menjawab permasalahan keamanan dan ketersediaan sistem.

    \item \textbf{Perancangan Sistem (\textit{System Design})} \\
    Pada tahap ini, spesifikasi kebutuhan diterjemahkan menjadi representasi desain perangkat lunak yang mencakup tiga fokus utama. Pertama, dilakukan pemodelan arsitektur sistem hibrida yang menggambarkan interaksi antara server penerbit, penyimpanan lokal (\textit{local cache}), dan perangkat pemindai. Kedua, dilakukan perancangan logika melalui diagram alur (\textit{Flowchart}) untuk mendetailkan algoritma pembangkitan tiket, mekanisme validasi mandiri (\textit{offline}), dan protokol sinkronisasi data asinkron. Terakhir, tahap ini meliputi perancangan antarmuka pengguna (\textit{User Interface}) untuk memastikan fitur pemindaian dapat berjalan responsif.

    \item \textbf{Implementasi (\textit{Implementation})} \\
    Tahapan ini bertujuan untuk merealisasikan rancangan desain menjadi unit program yang fungsional. Implementasi dilakukan dengan mengembangkan aplikasi seluler (\textit{mobile app}) menggunakan teknologi yang relevan (seperti React Native/Expo) yang berfungsi sebagai dompet tiket dan alat pemindai. Logika keamanan inti akan diimplementasikan untuk menangani dua peran: (1) Sisi \textit{Backend} untuk penandatanganan dan penerbitan tiket, dan (2) Sisi Perangkat Pemindai untuk melakukan dekripsi dan validasi tanda tangan digital secara lokal tanpa ketergantungan penuh pada koneksi server.

    \item \textbf{Pengujian (\textit{Testing})} \\
    Setelah prototipe berhasil dibangun, tahap pengujian dilakukan untuk memverifikasi keandalan sistem. Pengujian mencakup dua skenario utama: (1) \textit{Security Testing} untuk menguji ketahanan terhadap serangan tangkapan layar (\textit{screenshot}) dan pemalsuan tiket; serta (2) \textit{Availability Testing} untuk menguji kemampuan sistem melakukan validasi tiket dalam kondisi tanpa koneksi internet (\textit{offline}) dan memastikan data tersinkronisasi dengan benar saat koneksi kembali tersedia (\textit{batching}).

    \item \textbf{Operasi dan Pemeliharaan (\textit{Operation and Maintenance})} \\
    Dalam konteks pengerjaan tugas akhir, tahapan ini diadaptasi menjadi fase dokumentasi dan penyusunan laporan. Pengerjaannya difokuskan pada penyusunan laporan akhir. Seluruh artefak tugas akhir, mulai dari hasil analisis, desain, kode program, hingga hasil pengujian keamanan dan ketersediaan, akan didokumentasikan secara sistematis. Tahapan ini juga mencakup penarikan kesimpulan untuk menjawab rumusan masalah serta saran perbaikan untuk pengembangan selanjutnya.
\end{enumerate}