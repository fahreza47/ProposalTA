% ==========================================
% BAB I PENDAHULUAN
% ==========================================
\chapter{PENDAHULUAN}
\label{chap:pendahuluan}
% --- Latar Belakang ---
\section{Latar Belakang}

Berdasarkan Kamus Besar Bahasa Indonesia (KBBI), Tiket atau karcis adalah surat kecil (carik kertas khusus) sebagai tanda telah membayar ongkos dan sebagainya (untuk naik bus, menonton bioskop, dan sebagainya). Tiket merupakan sebuah dokumen yang berfungsi sebagai bukti hak akses atau tanda pembayaran yang sah untuk menggunakan suatu layanan atau memasuki suatu area tertentu. Secara historis, tiket konvensional dalam bentuk fisik telah menjadi bagian tak terpisahkan dari berbagai sektor, mulai dari transportasi hingga hiburan.  Namun, seiring dengan pesatnya perkembangan teknologi informasi, terjadi pergeseran paradigma menuju digitalisasi tiket menjadi tiket elektronik (\textit{e-ticket}). Inovasi layanan ini sangat erat kaitannya dengan adopsi sistem teknis berbasis komputer yang memungkinkan peningkatan efisiensi dan efektivitas operasional \cite{lubeck2012electronic}. Pergeseran paradigma tersebut didorong oleh kebutuhan untuk meningkatkan manajemen informasi yang sebelumnya sulit dilakukan dengan sistem manual atau kartu magnetik \cite{lubeck2012electronic}. 

Adopsi \textit{e-ticket} mulai marak pada awal tahun 2000-an, yang dipelopori oleh industri penerbangan di tahun 1990-an, dan kini telah diadopsi secara masif di berbagai sektor. \textit{E-ticket} menawarkan berbagai keunggulan signifikan dibandingkan tiket konvensional yang rentan terhadap inefisiensi. \textcite{lubeck2012electronic} menyoroti bahwa sistem konvensional seringkali terkendala oleh lemahnya kontrol operasional yang menyebabkan maraknya perdagangan tiket ilegal serta penyalahgunaan manfaat tiket khusus (seperti tiket pelajar) karena sulitnya identifikasi pengguna. Dari sisi pengguna, \textit{e-ticket} memberikan kemudahan distribusi dan akses, menghilangkan risiko kehilangan tiket fisik, serta membantu menghindari antrean panjang. Selain itu, sistem ini juga lebih efisien dari segi biaya operasional dan memiliki dampak lingkungan yang lebih rendah karena mengurangi penggunaan kertas \cite{chen2007passenger}.

Untuk merealisasikan berbagai keunggulan \textit{e-ticket} tersebut, diperlukan medium representasi data yang efisien dan kompatibel dengan perangkat pengguna. Di antara berbagai alternatif teknologi, \textit{Quick Response Code} (QR Code) muncul sebagai solusi dominan yang diadopsi secara luas dalam implementasi \textit{e-ticket}. QR Code adalah jenis kode batang (\textit{barcode}) matriks atau kode dua dimensi yang dapat menyimpan informasi digital \cite{shin2012psychology}. Tidak seperti \textit{barcode} satu dimensi, QR Code mengenkode data secara horizontal dan vertikal, menawarkan kepadatan informasi yang lebih tinggi dan kecepatan pembacaan yang lebih cepat \cite{alsuhibany2025innovative}. \textcite{tiwari2016introduction} menjelaskan bahwa tingkat penerimaan QR Code yang tinggi secara global berbanding lurus dengan pertumbuhan pengguna ponsel pintar, yang memungkinkan teknologi ini menjangkau konsumen secara luas dan cepat. Ubikuitas perangkat pemindai yang terintegrasi dalam ponsel pintar, menjadikan QR Code pilihan yang praktis dan efisien untuk diterapkan sebagai medium \textit{e-ticket}. Kepopuleran dan kemudahan akses tersebut mendorong adopsi luas QR Code pada gerbang transportasi maupun acara hiburan. Akan tetapi, di balik kenyamanan tersebut, model \textit{e-ticket} konvensional yang mengandalkan QR Code dalam bentuk statis, secara inheren mewarisi celah keamanan yang serius.

Sistem \textit{e-ticket} pada umumnya mengadopsi model QR Code statis. Pada model ini, data tiket seperti identitas pengguna atau tautan validasi, dienkode secara langsung ke dalam pola matriks citra. Karakteristik fundamental dari QR Code statis adalah informasi yang tersimpan di dalamnya bersifat tetap (\textit{fixed information}) \cite{yanuarafi2023perbandingan}; artinya, setelah kode dibangkitkan (\textit{generated}), pola visualnya tidak akan berubah dan terus valid sepanjang masa berlaku tiket. Proses validasi bergantung sepenuhnya pada pemindaian di pintu masuk, yaitu saat alat pemindai menerjemahkan kembali pola matriks menjadi data identitas untuk dicocokkan dengan basis data. Meskipun arsitektur ini menawarkan kemudahan implementasi, menurut \textcite{yanuarafi2023perbandingan}, penggunaan QR Code statis memiliki kelemahan signifikan dalam aspek keamanan. Sifatnya yang permanen membuat sistem ini rentan terhadap penyalahgunaan, seperti duplikasi ilegal dan pemalsuan, yang pada akhirnya mengancam integritas ekosistem \textit{e-ticket} secara keseluruhan.

Kelemahan mendasar dari arsitektur statis adalah sifatnya yang ``sekali terbit, berlaku selamanya'' tanpa mekanisme pembaruan autentikasi. Celah tersebut dieksploitasi secara luas melalui serangan penggandaan (\textit{cloning}) dan serangan putar ulang (\textit{replay attack}). \textcite{sung2015security} dalam analisis keamanannya menegaskan bahwa QR Code sangat mudah diduplikasi melalui fitur tangkapan layar (\textit{screen capture}) pada perangkat seluler, yang kemudian dapat ditransfer ke pihak lain tanpa bisa dicegah oleh sistem konvensional. Dampak dari kerentanan ini menciptakan efek domino kerusakan pada ekosistem pertiketan. Pertama, pada aspek validasi di lapangan, insiden konser Coldplay di Jakarta tahun 2023 memperlihatkan kekacauan di pintu masuk ketika banyak pemegang tiket sah gagal mendapatkan akses karena tiket mereka telah digandakan dan digunakan lebih dulu oleh pihak lain. Berdasarkan analisis hukum, modus ini terjadi karena pelaku mempelajari desain visual tiket statis lalu menggandakannya untuk dijual ke banyak korban \cite{berma2023analisis}. Kedua, lemahnya sistem keamanan turut menyuburkan praktik percaloan (\textit{scalping}), yaitu dengan menjual kembali tiket yang telah dibeli secara legal, dengan harga berkali-kali lipat dari harga resmi sehingga merusak kewajaran pasar \cite{Pamela2023}. Ketiga, kegagalan kontrol akses berlanjut hingga ke dalam arena, seperti pada salah satu pertandingan Timnas Indonesia di GBK. Pada kasus tersebut, penonton tanpa hak akses valid berhasil masuk dan menduduki kursi pemegang tiket sah, memicu konflik fisik dan ketidaknyamanan \cite{Kurniawan2024}. Terakhir, dari sisi kerugian materiil, investigasi Kompas mengungkapkan data Pusat Pelaporan dan Analisis Transaksi Keuangan (PPATK) yang mencatat 182 kasus transaksi mencurigakan terkait penipuan tiket konser pada tahun 2024 dengan total nilai Rp 2,3 miliar \cite{diveranta2025jejak}. Rangkaian kasus ini menegaskan bahwa sistem konvensional saat ini gagal memberikan perlindungan menyeluruh, baik dari sisi keamanan akses, keadilan harga, maupun perlindungan hak konsumen.

Kompleksitas permasalahan tersebut mulai dari kekacauan validasi fisik, inflasi harga akibat percaloan, hingga kerugian materiil akibat penipuan, membuktikan bahwa sistem verifikasi yang hanya mengandalkan QR Code statis tidak lagi memadai. Diperlukan sebuah pendekatan komprehensif untuk menjamin integritas transaksi dan data. Berdasarkan analisis masalah tersebut, sebuah sistem \textit{e-ticket} yang ideal harus memiliki tiga karakteristik pertahanan utama. Pertama, tiket harus bersifat dinamis (\textit{dynamic}) menggunakan mekanisme pembangkitan QR Code yang berubah secara berkala berbasis waktu, sehingga tangkapan layar menjadi tidak valid setelah durasi tertentu \cite{sung2015security}. Kedua, tiket harus menjamin kerahasiaan (\textit{confidentiality}) melalui enkripsi muatan data (\textit{payload}) untuk melindungi privasi pengguna dari pembacaan data sembarangan serta risiko eksfiltrasi data dari penyimpanan lokal \cite{sung2015security}. Ketiga, tiket harus bersifat aman (\textit{secure}) menggunakan mekanisme tanda tangan digital (\textit{digital signature}) yang menjamin aspek nirsangkal (\textit{non-repudiation}), untuk memastikan tiket diterbitkan oleh otoritas yang sah dan tidak dimodifikasi.

Oleh karena itu, penelitian ini bertujuan untuk merancang dan mengimplementasikan sistem \textit{e-ticket} yang mengusung konsep \textit{Dynamic Secure QR Code}. Urgensi penelitian ini difokuskan pada sektor hiburan dan olahraga skala besar, mengingat sektor ini memiliki risiko kerugian tertinggi akibat manipulasi tiket. Melalui implementasi sistem ini, diharapkan tercipta ekosistem pertiketan yang lebih sehat yang memberikan manfaat ganda: konsumen mendapatkan jaminan perlindungan hak akses dan data pribadi, sementara penyelenggara acara dapat memitigasi kebocoran pendapatan (\textit{revenue leakage}) akibat tiket palsu. Penelitian ini akan berfokus pada pengembangan prototipe sistem yang mampu membangkitkan dan memvalidasi tiket dengan arsitektur keamanan berlapis tersebut.

% --- Rumusan Masalah ---
\section{Rumusan Masalah}
Berdasarkan latar belakang yang telah diuraikan, teridentifikasi adanya kelemahan fundamental pada arsitektur \textit{e-ticket} berbasis QR Code statis yang rentan terhadap berbagai eksploitasi keamanan. Oleh karena itu, rumusan masalah dalam penelitian ini adalah sebagai berikut:
\begin{enumerate} 
\item Bagaimana merancang arsitektur sistem \textit{e-ticket} yang mengintegrasikan konsep \textit{Dynamic Secure QR Code} untuk menjamin aspek kerahasiaan (\textit{confidentiality}), integritas (\textit{integrity}), dan nirsangkal (\textit{non-repudiation})? 
\item Bagaimana mekanisme pembangkitan dan validasi tiket menggunakan kombinasi algoritma enkripsi, pembangkitan kode dinamis berbasis waktu, dan Tanda Tangan Digital untuk mencegah pemalsuan dan modifikasi tiket.
\item Bagaimana efektivitas penerapan kode dinamis berbasis waktu dalam memitigasi serangan penggandaan tiket (\textit{cloning}) melalui tangkapan layar (\textit{screenshot}) dan serangan putar ulang (\textit{replay attack}) dibandingkan dengan sistem statis? 
\end{enumerate}

% --- Tujuan ---
\section{Tujuan}
Mengacu pada rumusan masalah yang telah dipaparkan, tujuan utama dari penelitian ini adalah:
\begin{enumerate}
\item Merancang arsitektur sistem \textit{e-ticket} yang mampu memenuhi standar keamanan informasi, meliputi aspek kerahasiaan data (\textit{confidentiality}), integritas data (\textit{integrity}), dan nirsangkal (\textit{non-repudiation}).
\item Mengimplementasikan prototipe (\textit{proof-of-concept}) sistem yang dapat membangkitkan dan memvalidasi tiket menggunakan kombinasi enkripsi muatan, kode dinamis berbasis waktu, dan tanda tangan digital (\textit{digital signature}).
\item Mengevaluasi efektivitas sistem yang diusulkan melalui serangkaian pengujian keamanan untuk membuktikan kemampuannya dalam memitigasi serangan penggandaan tiket (\textit{cloning}) dan pemalsuan tiket (\textit{forgery}).
\end{enumerate}

% --- Batasan Masalah ---
\section{Batasan Masalah}
Agar pengerjaan tugas akhir dapat lebih terarah dan tidak melenceng dari tujuan utamanya, ruang lingkup permasalahan dibatasi sebagai berikut:
\begin{enumerate}
\item Penelitian ini berfokus pada perancangan dan implementasi modul inti keamanan, yaitu proses pembangkitan (\textit{generation}) dan validasi (\textit{validation}) \textit{Dynamic Secure QR Code}, tanpa membahas aspek antarmuka pengguna (UI/UX) secara mendalam.
\item Penelitian ini tidak akan membangun sistem \textit{e-commerce} atau \textit{marketplace} penjualan tiket yang utuh. Fitur pendukung seperti manajemen akun pengguna, gerbang pembayaran (\textit{payment gateway}), dan manajemen acara (\textit{event management}) berada di luar lingkup penelitian.
\item Luaran sistem yang dibangun berupa prototipe (\textit{proof-of-concept}) yang bertujuan untuk mendemonstrasikan kelayakan logika keamanan, bukan sebagai aplikasi skala produksi yang siap dirilis secara komersial (siap pakai).
\item Implementasi teknis prototipe akan dikembangkan menggunakan bahasa pemrograman Python dengan memanfaatkan pustaka (\textit{library}) kriptografi standar dan modul QR Code yang relevan.
\item Penelitian tidak mencakup perancangan perangkat keras (\textit{hardware}) pemindai khusus. Proses pemindaian dan validasi diasumsikan dilakukan menggunakan perangkat lunak pada ponsel pintar berbasis kamera.
\end{enumerate}

% --- Metodologi ---
\section{Metodologi}
Pengerjaan tugas akhir ini menerapkan kerangka kerja \textit{Software Development Life Cycle} (SDLC) dengan pendekatan model \textit{Waterfall} sebagai metodologi. Model ini dipilih karena pengerjaan tugas akhir yang memiliki kebutuhan sistem (\textit{requirements}) yang didefinisikan secara jelas di tahap awal, yaitu berfokus pada aspek keamanan QR Code, serta membutuhkan alur pengerjaan yang terstruktur. Tahapan pengembangan sistem dalam pengerjaan tugas akhir mengacu pada standar rekayasa perangkat lunak menurut \textcite{sommerville2016software}, yang secara visual dapat dilihat pada Gambar \ref{fig:metodologi_waterfall}.

\begin{figure}[h]
    \centering
    \captionsetup{justification=centering}
    	\includegraphics[width=0.9\textwidth]{image/waterfall.png}
    \caption{Alur Metodologi Penelitian Model Waterfall}
    \label{fig:metodologi_waterfall}
\end{figure}

Rincian tahapan yang akan dilalui selama pelaksanaan tugas akhir adalah sebagai berikut:

\begin{enumerate}
    \item \textbf{Analisis Kebutuhan (\textit{Requirements Analysis})} \\
    Tahapan ini merupakan langkah fundamental untuk mengumpulkan fakta empiris dan merumuskan spesifikasi kebutuhan sistem. Proses investigasi dilakukan dengan mengobservasi fenomena kegagalan sistem \textit{e-ticket} pada acara berskala besar di media sosial, serta mengumpulkan data sekunder dari sumber kredibel, seperti laporan PPATK dan pemberitaan media massa terkait modus kejahatan tiket. Selain itu, dilakukan studi literatur terhadap penelitian terdahulu dan standar teknis terkait algoritma kriptografi untuk menentukan kombinasi teknologi yang tepat, seperti mekanisme \textit{Time-based One-Time Password} (TOTP) dan Tanda Tangan Digital, untuk menjawab permasalahan keamanan yang telah dirumuskan.

    \item \textbf{Perancangan Sistem (\textit{System Design})} \\
    Pada tahap ini, spesifikasi kebutuhan diterjemahkan menjadi representasi desain perangkat lunak yang mencakup tiga fokus utama. Pertama, dilakukan pemodelan arsitektur sistem dengan merancang diagram arsitektur yang menggambarkan interaksi antara sisi klien (aplikasi seluler) dan sisi server (\textit{backend}). Kedua, dilakukan perancangan logika dan alur data melalui pembuatan diagram alur (\textit{Flowchart}) dan diagram aktivitas (\textit{Activity Diagram}) untuk mendetailkan algoritma pembangkitan tiket yang melibatkan proses enkripsi \textit{payload} dan penandatanganan digital. Terakhir, tahap ini meliputi perancangan antarmuka pengguna (\textit{User Interface}) untuk aplikasi seluler guna memastikan fitur pemindaian dan tampilan tiket dapat digunakan dengan baik.

    \item \textbf{Implementasi (\textit{Implementation})} \\
    Tahapan ini bertujuan untuk merealisasikan rancangan desain menjadi unit program yang fungsional. Implementasi dilakukan dengan mengembangkan aplikasi seluler (\textit{mobile app}) menggunakan kerangka kerja \textbf{React Native/Expo} yang berfungsi sebagai antarmuka pengguna dan alat pemindai QR Code. Aplikasi ini akan terintegrasi dengan logika keamanan inti yang dibangun menggunakan bahasa pemrograman \textbf{Python}, yang bertugas menangani proses kriptografi, pembangkitan token dinamis, dan validasi tanda tangan digital di sisi \textit{backend}.

    \item \textbf{Pengujian (\textit{Testing})} \\
    Setelah prototipe berhasil dibangun, tahap pengujian dilakukan untuk memverifikasi keandalan sistem dan memastikannya bebas dari cacat logika keamanan. Pengujian akan dilakukan menggunakan skenario \textit{Security Testing} yang mensimulasikan serangan nyata, seperti uji ketahanan terhadap serangan tangkapan layar (\textit{screenshot}) dan uji deteksi pemalsuan tiket. Tujuannya adalah untuk membuktikan secara empiris bahwa sistem mampu menolak tiket yang tidak sah atau tiket yang telah dimodifikasi.

    \item \textbf{Operasi dan Pemeliharaan (\textit{Operation and Maintenance})} \\
    Dalam konteks pengerjaan tugas akhir, tahapan ini diadaptasi menjadi fase dokumentasi dan penyusunan laporan. Pengerjaannya difokuskan pada penyusunan laporan akhir. Seluruh artefak tugas akhir, mulai dari hasil analisis, desain, kode program, hingga hasil pengujian, akan didokumentasikan secara sistematis. Tahapan ini juga mencakup penarikan kesimpulan berdasarkan hasil pengujian untuk menjawab rumusan masalah yang telah ditetapkan di awal penelitian serta saran perbaikan untuk pengembangan selanjutnya.
\end{enumerate}